\documentclass[11pt, a4paper]{article}
\usepackage[utf8]{inputenc}
\usepackage{amsmath, amssymb, amsfonts}
\usepackage{geometry}
\usepackage{ctex} % 用于中文显示
\newtheorem{lt}{}
\newtheorem{solve}{解}
\newtheorem{sign}{注}

\geometry{left=2cm, right=2cm, top=2.5cm, bottom=2.5cm}

\begin{document}

\begin{center}
    {\LARGE \textbf{武汉理工大学研究生考试试卷(A 卷)}} \\
    \vspace{0.5cm}
    \textbf{2023 ~ 2024 学年 1 学期 \underline{\quad 矩阵论(学硕) \quad} 课程 \quad 2024 年 1 月 7 日} \\
    (请在答题本上作答,不必抄题,但须标明题目序号)
\end{center}

\vspace{0.5cm}

\section*{一、填空题(每小题 3 分,共 15 分)}
\begin{lt}
A为$m\times n$阶矩阵,那么$A^T$为$n\times m$阶矩阵,则$A^TA$为$n\times n$阶矩阵。由于$rankA=r$,因此解空间维数为$n-r$.
\end{lt}

\begin{lt}
$A^HA=\begin{bmatrix}
3&3\\3&9
\end{bmatrix}$,
$\lambda I-A^HA=\begin{bmatrix}
\lambda-3&-3\\-3&\lambda-9
\end{bmatrix}$,$\lambda = 6\pm2\sqrt{3}$,所以奇异值为$\sqrt{6\pm2\sqrt{3}}$。
\end{lt}

\begin{lt}
取$\beta_1=\alpha_1=\begin{bmatrix}
1\\1\\1\end{bmatrix}$,则$\gamma_1=\displaystyle\frac{\beta_1}{||\beta_1||}=\displaystyle\frac{1}{\sqrt3}\begin{bmatrix}
1\\1\\1\end{bmatrix}$.\\

取$\beta_2=\alpha_2-\displaystyle\frac{(\alpha_2, \beta_1)}{||\beta_1||^2}\beta_1=\begin{bmatrix}
2\\-1\\2\end{bmatrix}-\displaystyle\frac{3}{3}\begin{bmatrix}
1\\1\\1\end{bmatrix}=\begin{bmatrix}
1\\-2\\1\end{bmatrix}$,则$\gamma_2=\displaystyle\frac{\beta_2}{||\beta_2||}=\displaystyle\frac{1}{\sqrt6}\begin{bmatrix}
1\\-2\\1\end{bmatrix}$.\\

$$A=QR=(\gamma_1, \gamma_2)\begin{bmatrix}
||\beta_1||&(\alpha_2, \gamma_1)\\0&||\beta_2||
\end{bmatrix}
=\begin{bmatrix}
\displaystyle\frac{1}{\sqrt3}&\displaystyle\frac{1}{\sqrt6}\\\displaystyle\frac{1}{\sqrt3}&\displaystyle\frac{2}{\sqrt6}\\\displaystyle\frac{1}{\sqrt3}&\displaystyle\frac{1}{\sqrt6}\end{bmatrix}
\begin{bmatrix}
\sqrt3&\sqrt3\\0&\sqrt6\end{bmatrix}$$
\end{lt}


\begin{lt}\
矩阵范数 $\|A\|_1 = \max \displaystyle\sum_{i=1}^3|a_{ij}|=4$.

矩阵范数 $\|A\|_\infty = \max \displaystyle\sum_{j=1}^3|a_{ij}|=4$.

矩阵范数 $\|A\|_F = \sqrt{\displaystyle\sum_{i=1}^3\sum_{j=1}^3|a_{ij}|^2}=4$.
\end{lt}


\begin{lt}
$\lambda I-A = \begin{bmatrix}
\lambda+2&0&0\\
0&\lambda-i&1\\
0&1&\lambda-3i
\end{bmatrix}
\rightarrow
\begin{bmatrix}
1&0&\lambda-3i\\
\lambda-i&0&1\\
0&\lambda+2&0
\end{bmatrix}
\rightarrow
\begin{bmatrix}
1&0&\lambda-3i\\
0&0&(\lambda-2i)^2\\
0&\lambda+2&0
\end{bmatrix}\\
\rightarrow
\begin{bmatrix}
1&0&0\\
0&0&(\lambda-2i)^2\\
0&\lambda+2&0
\end{bmatrix}
\rightarrow
\begin{bmatrix}
1&0&0\\
0&\lambda+2&0\\
0&0&(\lambda-2i)^2
\end{bmatrix}$.
所以极小多项式$m_A(\lambda)=(\lambda+2)(\lambda-2i)^2$.
\end{lt}

\section*{二、(15 分)}
\textbf{解}:

$\lambda E - A=\begin{bmatrix} \lambda-2 & 0 & 0 \\ -1 & \lambda-1 & -1 \\ -1 & 1 & \lambda-3 \end{bmatrix}
\rightarrow
\begin{bmatrix} -1 & 1 & \lambda-3\\-1 & \lambda-1 & -1  \\ \lambda-2 & 0 & 0  \end{bmatrix}
\rightarrow
\begin{bmatrix} 1 & -1 & \lambda-3\\\lambda-1 & -1 & -1  \\  0&\lambda-2  & 0  \end{bmatrix}\\
\rightarrow
\begin{bmatrix} 1 & -1 & \lambda-3\\0 & \lambda-2 & -(\lambda-2)^2   \\  0&\lambda-2  & 0  \end{bmatrix}
\rightarrow
\begin{bmatrix} 1 & 0 & 0\\0 & \lambda-2 & -(\lambda-2)^2   \\  0&\lambda-2  & 0  \end{bmatrix}
\rightarrow
\begin{bmatrix} 1 & 0 & 0\\0 & \lambda-2 & 0 \\  0&0  & (\lambda-2)^2  \end{bmatrix}$,即$Smith$标准型。\\

由Smith标准型知特征值为$\lambda=2$(三重),行列式因子为:$D_1 = 1, D_2= \lambda-2, D_3 = (\lambda-2)^3$,\\不变因子为:$d_1 = 1, d_2= \lambda-2, d_3 = (\lambda-2)^2$,初等因子为$\lambda-2, (\lambda-2)^2$。\\

因此Jordan标准型为$\begin{bmatrix}
2&   &  \\
 & 2 & 1\\
 &   & 2\\
\end{bmatrix}= diag\left\{2, \begin{bmatrix}2 & 1\\2 & 0\end{bmatrix} \right\}$.\\

显然极小多项式$m_A(\lambda) = d_3 = (\lambda-2)^2$.




\section*{三、(15 分)}

\textbf{解}:(1)证明:映射 $T$ 对$F[t]_3$加法和数乘封闭,且包含$\mathbf{0}$向量,所以为线性变换.


(2)下面用两种方法求解,其中解法一为线性变换,解法二为基坐标。

\begin{solve}由已知$\{e_1, e_2, e_3\}=\{1, t, t^2\}$为一组基。

取$a=1, b=0, c=0$,有$T[f(t)]=T[1]=T[e_1]=1+2t+3t^2=
\begin{bmatrix}1&t&t^2\end{bmatrix}\begin{bmatrix}1\\2\\3\end{bmatrix}$.

取$a=0, b=1, c=0$,有$T[f(t)]=T[t]=T[e_2]=1+2t=
\begin{bmatrix}1&t&t^2\end{bmatrix}\begin{bmatrix}1\\2\\0\end{bmatrix}$.

取$a=0, b=0, c=3$,有$T[f(t)]=T[t^2]=T[e_3]=4+3t=
\begin{bmatrix}1&t&t^2\end{bmatrix}\begin{bmatrix}4\\3\\0\end{bmatrix}$.\\

所以$T(1, t, t^2) = T(e_1, e_2, e_3) =\left( T(e_1), T(e_2), T(e_3)\right)=(1, t, t^2)\begin{bmatrix}1&1&4\\2&2&3\\3&0&0\end{bmatrix}$.

不难看出$T$ 在基 $1, t, t^2$ 基下的表示矩阵为$\begin{bmatrix}1&1&4\\2&2&3\\3&0&0\end{bmatrix}$.
\end{solve}

\begin{solve}
$T[f(t)] = (a + b + 4c) + (2a + 2b + 3c)t + 3at^2=\begin{bmatrix}1&t&t^2\end{bmatrix}\begin{bmatrix}1&1&4\\2&2&3\\3&0&0\end{bmatrix}\begin{bmatrix}a\\b\\c\end{bmatrix}$.

显然$T$ 在基 $1, t, t^2$ 基下的表示矩阵为$\begin{bmatrix}1&1&4\\2&2&3\\3&0&0\end{bmatrix}$.\\
\end{solve}


(3)由于$rank\begin{bmatrix}1&1&4\\2&2&3\\3&0&0\end{bmatrix}=3$,所以Im$T=3$,显然可以取一组基为$\{1, t, t^2\}$.



\section*{四、(15 分)}
\textbf{解}:(1)先求出Smith标准型以得到极小多项式,再利用待定系数法计算即可。\\

$\lambda I-A = \begin{bmatrix} \lambda-2 & 0 & 0 \\ -1 & \lambda-1 & -1 \\ -1 & 1 & \lambda-3 \end{bmatrix}
\rightarrow
\begin{bmatrix} -1 & 1 & \lambda-3\\-1 & \lambda-1 & -1  \\ \lambda-2 & 0 & 0  \end{bmatrix}
\rightarrow
\begin{bmatrix} 1 & -1 & \lambda-3\\\lambda-1 & -1 & -1  \\  0&\lambda-2  & 0  \end{bmatrix}\\
\rightarrow
\begin{bmatrix} 1 & -1 & \lambda-3\\0 & \lambda-2 & -(\lambda-2)^2   \\  0&\lambda-2  & 0  \end{bmatrix}
\rightarrow
\begin{bmatrix} 1 & 0 & 0\\0 & \lambda-2 & -(\lambda-2)^2   \\  0&\lambda-2  & 0  \end{bmatrix}
\rightarrow
\begin{bmatrix} 1 & 0 & 0\\0 & \lambda-2 & 0 \\  0&0  & (\lambda-2)^2  \end{bmatrix}$.\\

显然$A$的极小多项式为$m_A(\lambda)=(\lambda-2)^2 $,deg\ $m_A(\lambda)$=2,所以待定系数的最高次为$2-1=1$次。设$f(\lambda) = a_0+a_1\lambda $。记$e^{\lambda} = g(\lambda)$.

考虑$m_A(\lambda) = (\lambda-2)^2 $,特征值为$\lambda = 2$(两重),带入$f(\lambda)$得:
$$f(2)=a_0+2a_1=g(2)=e^2$$
$$f'(2)=a_1=g'(2)=e^2$$

解上述方程组得到系数:
$$a_0= -e^2$$
$$a_1= e^2$$

所以$e^A=a_0E+a_1A=(A-E)\times e^2=\begin{bmatrix} 1 & 0 & 0\\1 & 0 & 1 \\  1&-1  & 2  \end{bmatrix}e^2$.

同理可得$e^{tA}=\begin{bmatrix} 1 & 0 & 0\\t & 1-t & t \\  t&-t  & 1+t \end{bmatrix}e^{2t}$.

(2)带入公式即可:
$$x(t)=e^{tA}x(0)=\begin{bmatrix} 1 & 0 & 0\\t & 1-t & t \\  t&-t  & 1+t \end{bmatrix}\begin{bmatrix} 1 \\ 0 \\ -1 \end{bmatrix}e^{2t}=\begin{bmatrix} 1 \\ 0 \\ -1 \end{bmatrix}e^{2t}$$


\section*{五、(15 分)}
\textbf{解}:(1)显然.

(2)由题意,$W$中向量为$AX = 0$的解空间,又注意到rank$A=2$,所以解空间维数为2。\\

解方程$AX = 0$得$X = c_1\begin{bmatrix} 1 \\ -1 \\ -1\\ 1\end{bmatrix}+c_2\begin{bmatrix} 1 \\ 1 \\ 1\\ 1\end{bmatrix}(c_1,c_2\in \mathbf{R})$,二者恰好正交。\\

显然$W$的一组标准正交基可取$\displaystyle\frac{1}{2}\begin{bmatrix} 1 \\ -1 \\ -1\\ 1\end{bmatrix}$和$\displaystyle\frac{1}{2}\begin{bmatrix} 1 \\ 1 \\ 1\\ 1\end{bmatrix}$。

\begin{sign}
若上面解出来的$X$两组解向量并不正交,进行一次Gram-Schmidt正交化即可。
\end{sign}



\section*{六、(25 分)}
\textbf{解}:(1)注意到矩阵A的一、二列相同,且二三列显然线性无关,因此rankA=2。\\

利用矩阵乘法的右乘有$A = \begin{bmatrix} 0 & 0 & 2 \\ 1 & 1 & 0 \\ 0 & 0 & 1 \\ 1 & 1 & 1 \end{bmatrix}=BC= \begin{bmatrix}0 & 2 \\1 & 0 \\0 & 1 \\1 & 1 \end{bmatrix}\begin{bmatrix}1&1&0\\0&0&1\end{bmatrix}$.\\

即得满秩分解A=BC,其中:

$$B= \begin{bmatrix}0 & 2 \\1 & 0 \\0 & 1 \\1 & 1 \end{bmatrix},
C=\begin{bmatrix}1&1&0\\0&0&1\end{bmatrix}$$

(2)由于A为$4\times3$阶矩阵,所以Moore-Penrose广义逆$A^+$应为$3\times4$阶矩阵,直接带入公式即可。

$$\begin{aligned}A^+ &= C^H(CC^H)^{-1}(BB^H)^{-1}B^H\\
&= \begin{bmatrix}1&0\\1&0\\0&1\end{bmatrix}
\begin{bmatrix}2 & 0\\0 & 1\end{bmatrix}^{-1}
\begin{bmatrix}2 & 1\\1 & 6\end{bmatrix}^{-1}
\begin{bmatrix}0&1&0&1\\2&0&1&1\end{bmatrix}\\
&=\displaystyle\frac{1}{22}\begin{bmatrix}1&0\\1&0\\0&1\end{bmatrix}
\begin{bmatrix}1 & 0\\0 & 2\end{bmatrix}
\begin{bmatrix}6 & -1\\-1 & 2\end{bmatrix}
\begin{bmatrix}0&1&0&1\\2&0&1&1\end{bmatrix}
=\displaystyle\frac{1}{22}
\begin{bmatrix}
-2&6&-1&5\\
-2&6&-1&5\\
8&-2&4&2\\
\end{bmatrix}
\end{aligned}$$

$$A^+=\displaystyle\frac{1}{22}
\begin{bmatrix}
-2&6&-1&5\\
-2&6&-1&5\\
8&-2&4&2\\
\end{bmatrix}$$


(3)相容性即验证$AA^+b$是否与b相等即可。

$$AA^+ = \displaystyle\frac{1}{22}
\begin{bmatrix} 0 & 0 & 2 \\ 1 & 1 & 0 \\ 0 & 0 & 1 \\ 1 & 1 & 1 \end{bmatrix}
\begin{bmatrix}
-2&6&-1&5\\
-2&6&-1&5\\
8&-2&4&2\\
\end{bmatrix}=\displaystyle\frac{1}{22}
\begin{bmatrix}
16&-4&8&4\\
-4&12&-2&10\\
8&-2&4&2\\
4&10&2&12
\end{bmatrix}
$$
$$AA^+b=\displaystyle\frac{1}{22}
\begin{bmatrix}
16&-4&8&4\\
-4&12&-2&10\\
8&-2&4&2\\
4&10&2&12
\end{bmatrix}
\begin{bmatrix} 1 \\ 1 \\ 1 \\ 1 \end{bmatrix}
=\displaystyle\frac{1}{11}\begin{bmatrix}12\\8\\6\\14\end{bmatrix}\neq\begin{bmatrix}1\\1\\1\\1\end{bmatrix}=b$$

显然,方程组 \( Ax = b \) 的相容性不相容。

(5)极小范数最小二乘解为$x^*=A^+b$。

$$x^*=A^+b=\displaystyle\frac{1}{22}
\begin{bmatrix}
-2&6&-1&5\\
-2&6&-1&5\\
8&-2&4&2\\
\end{bmatrix}
\begin{bmatrix}1\\1\\1\\1\end{bmatrix}
=\displaystyle\frac{1}{11}
\begin{bmatrix}4\\4\\6\end{bmatrix}$$

(4)最小二乘解为$x=A^+b+(E-A^+A)y$,其中$y\in \mathbf{R^3}$。
$$A^+A=\displaystyle\frac{1}{22}
\begin{bmatrix}
-2&6&-1&5\\
-2&6&-1&5\\
8&-2&4&2\\
\end{bmatrix}
\begin{bmatrix} 0 & 0 & 2 \\ 1 & 1 & 0 \\ 0 & 0 & 1 \\ 1 & 1 & 1 \end{bmatrix}
=\displaystyle\frac{1}{2}
\begin{bmatrix} 1 & 1 & 0 \\1 & 1 &0\\0 & 0 &2 \end{bmatrix} 
$$

$$E-A^+A=\displaystyle\frac{1}{2}
\begin{bmatrix} 1 & -1 & 0\\ -1 & 1 &0\\0 & 0 & 0 \end{bmatrix} $$

取$y = \begin{bmatrix} a & b & c \end{bmatrix}^T$,代入$(E-A^+A)y$得:
$$(E-A^+A)y=\displaystyle\frac{1}{2}
\begin{bmatrix} 1 & -1 & 0\\ -1 & 1 &0\\0 & 0 & 0 \end{bmatrix}\begin{bmatrix} a \\ b \\ c \end{bmatrix}=
\begin{bmatrix} a-b \\ b-a \\ 0 \end{bmatrix}
=(a-b)\begin{bmatrix} 1 \\ -1 \\ 0 \end{bmatrix}
$$

由于$y$是任意的,因此$(E-A^+A)y=c\begin{bmatrix} 1 \\ -1 \\ 0 \end{bmatrix},(c\in \mathbf{R})$.\\

综上,最小二乘解为$x=\displaystyle\frac{1}{11}
\begin{bmatrix}4\\4\\6\end{bmatrix}+c\begin{bmatrix} 1 \\ -1 \\ 0 \end{bmatrix},(c\in \mathbf{R})$.


\end{document}