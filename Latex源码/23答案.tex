\documentclass[11pt, a4paper]{article}
\usepackage[utf8]{inputenc}
\usepackage{amsmath, amssymb, amsfonts}
\usepackage{geometry}
\usepackage{ctex} % 用于中文显示
\newtheorem{lt}{}
\geometry{left=2cm, right=2cm, top=2.5cm, bottom=2.5cm}

\begin{document}

\begin{center}
    {\LARGE \textbf{武汉理工大学研究生考试试卷(A 卷)}} \\
    \vspace{0.5cm}
    \textbf{2023 ~ 2024 学年 1 学期 \underline{\quad 矩阵论(学硕) \quad} 课程 \quad 2024 年 1 月 7 日} \\
    (请在答题本上作答,不必抄题,但须标明题目序号)
\end{center}

\vspace{0.5cm}

\section*{一、填空题(每小题 3 分,共 15 分)}
\begin{lt}
A为$m\times n$阶矩阵,那么$A^T$为$n\times m$阶矩阵,则$A^TA$为$n\times n$阶矩阵。由于$rankA=r$,因此解空间维数为$n-r$.
\end{lt}

\begin{lt}
$A^HA=\begin{bmatrix}
3&3\\3&9
\end{bmatrix}$,
$\lambda I-A^HA=\begin{bmatrix}
\lambda-3&-3\\-3&\lambda-9
\end{bmatrix}$,$\lambda = 6\pm2\sqrt{3}$,所以奇异值为$\sqrt{6\pm2\sqrt{3}}$。
\end{lt}

\begin{lt}
取$\beta_1=\alpha_1=\begin{bmatrix}
1\\1\\1\end{bmatrix}$,则$\gamma_1=\displaystyle\frac{\beta_1}{||\beta_1||}=\displaystyle\frac{1}{\sqrt3}\begin{bmatrix}
1\\1\\1\end{bmatrix}$.\\

取$\beta_2=\alpha_2-\displaystyle\frac{(\alpha_2, \beta_1)}{||\beta_1||^2}\beta_1=\begin{bmatrix}
2\\-1\\2\end{bmatrix}-\displaystyle\frac{3}{3}\begin{bmatrix}
1\\1\\1\end{bmatrix}=\begin{bmatrix}
1\\-2\\1\end{bmatrix}$,则$\gamma_2=\displaystyle\frac{\beta_2}{||\beta_2||}=\displaystyle\frac{1}{\sqrt6}\begin{bmatrix}
1\\-2\\1\end{bmatrix}$.\\

$$A=QR=(\gamma_1, \gamma_2)\begin{bmatrix}
||\beta_1||&(\alpha_2, \gamma_1)\\0&||\beta_2||
\end{bmatrix}
=\begin{bmatrix}
\displaystyle\frac{1}{\sqrt3}&\displaystyle\frac{1}{\sqrt6}\\\displaystyle\frac{1}{\sqrt3}&\displaystyle\frac{2}{\sqrt6}\\\displaystyle\frac{1}{\sqrt3}&\displaystyle\frac{1}{\sqrt6}\end{bmatrix}
\begin{bmatrix}
\sqrt3&\sqrt3\\0&\sqrt6\end{bmatrix}$$
\end{lt}


\begin{lt}\
矩阵范数 $\|A\|_1 = \max \displaystyle\sum_{i=1}^3|a_{ij}|=4$.

矩阵范数 $\|A\|_\infty = \max \displaystyle\sum_{j=1}^3|a_{ij}|=4$.

矩阵范数 $\|A\|_F = \sqrt{\displaystyle\sum_{i=1}^3\sum_{j=1}^3|a_{ij}|^2}=4$.
\end{lt}


\begin{lt}
$\lambda I-A = \begin{bmatrix}
\lambda+2&0&0\\
0&\lambda-i&1\\
0&1&\lambda-3i
\end{bmatrix}
\rightarrow
\begin{bmatrix}
1&0&\lambda-3i\\
\lambda-i&0&1\\
0&\lambda+2&0
\end{bmatrix}
\rightarrow
\begin{bmatrix}
1&0&\lambda-3i\\
0&0&(\lambda-2i)^2\\
0&\lambda+2&0
\end{bmatrix}\\
\rightarrow
\begin{bmatrix}
1&0&0\\
0&0&(\lambda-2i)^2\\
0&\lambda+2&0
\end{bmatrix}
\rightarrow
\begin{bmatrix}
1&0&0\\
0&\lambda+2&0\\
0&0&(\lambda-2i)^2
\end{bmatrix}$.
所以极小多项式$m_A(\lambda)=(\lambda+2)(\lambda-2i)^2$.
\end{lt}

\section*{二、(15 分)}
设 $A = \begin{bmatrix} 2 & 0 & 0 \\ 1 & 1 & 1 \\ 1 & -1 & 3 \end{bmatrix}$
\begin{enumerate}
    \item 求 $A$ 的行列式因子,不变因子,初等因子;
    \item 求 $A$ 的 Jordan 标准形和 $\lambda E - A$ 的 Smith 标准形;
    \item 求 $A$ 的最小多项式。
\end{enumerate}

$\lambda E - A=\begin{bmatrix} \lambda-2 & 0 & 0 \\ -1 & \lambda-1 & -1 \\ -1 & 1 & \lambda-3 \end{bmatrix}
\rightarrow
\begin{bmatrix} -1 & 1 & \lambda-3\\-1 & \lambda-1 & -1  \\ \lambda-2 & 0 & 0  \end{bmatrix}
\rightarrow
\begin{bmatrix} 1 & -1 & \lambda-3\\\lambda-1 & -1 & -1  \\  0&\lambda-2  & 0  \end{bmatrix}\\
\rightarrow
\begin{bmatrix} 1 & -1 & \lambda-3\\0 & \lambda-2 & -(\lambda-2)^2   \\  0&\lambda-2  & 0  \end{bmatrix}
\rightarrow
\begin{bmatrix} 1 & 0 & 0\\0 & \lambda-2 & -(\lambda-2)^2   \\  0&\lambda-2  & 0  \end{bmatrix}
\rightarrow
\begin{bmatrix} 1 & 0 & 0\\0 & \lambda-2 & 0 \\  0&0  & (\lambda-2)^2  \end{bmatrix}$,即$Smith$标准型。

\section*{三、(15 分)}设 $F[t]_3 = \{f(t) = a + bt + ct^2 \mid a, b, c \in \mathbb{R}\}$,对 $F[t]_3$ 中的任意元素
$f(t) = a + bt + ct^2$,定义映射 $T[f(t)] = (a + b + 4c) + (2a + 2b + 3c)t + 3at^2$。\\
\begin{enumerate}
    \item 证明 $T$ 是 $F[t]_3$ 上的线性变换;
    \item 求 $T$ 在基 $1, t, t^2$ 基下的矩阵;
    \item 求 $T$ 的像空间 $\mathrm{Im} T$ 的一组基和维数。
\end{enumerate}

\vspace{0.5cm}

\section*{四、(15 分)}
已知微分方程组\[
\begin{cases} 
\dfrac{dx(t)}{dt} = Ax(t) \\ 
x(0) = x_0 
\end{cases}, \quad A = \begin{bmatrix} 2 & 0 & 0 \\ 1 & 1 & 1 \\ 1 & -1 & 3 \end{bmatrix}, \quad x_0 = \begin{bmatrix} 1 \\ 0 \\ -1 \end{bmatrix}
\]
\begin{enumerate}
    \item 求 $e^{At}, e^A$;
    \item 求微分方程组的解。
\end{enumerate}

\vspace{0.5cm}

\section*{五、(15 分)}设 $\mathbb{R}^4$ 的子空间
\[ W = \{ X = (x_1, x_2, x_3, x_4)^T \mid AX = 0 \}, \text{ 其中 } A = \begin{bmatrix} 1 & 1 & -1 & -1 \\ -1 & 1 & 1 & -1 \\ -1 & -1 & 1 & 1 \\ 1 & -1 & -1 & 1 \end{bmatrix} \]
对于任意的 $X = (x_1, x_2, x_3, x_4)^T, Y = (y_1, y_2, y_3, y_4)^T \in W$,定义内积
\[ (X, Y) = x_1y_1 + 2x_2y_2 + x_3y_3 + 2x_4y_4 \]
\begin{enumerate}
    \item 证明 $W$ 为子空间;
    \item 求 $W$ 一组标准正交基。
\end{enumerate}

\vspace{0.5cm}

\section*{六、(25 分)}
设 $A = \begin{bmatrix} 0 & 0 & 2 \\ 1 & 1 & 0 \\ 0 & 0 & 1 \\ 1 & 1 & 1 \end{bmatrix}, b = \begin{bmatrix} 1 \\ 1 \\ 1 \\ 1 \end{bmatrix}$

\begin{enumerate}
    \item 求 $A$ 的满秩分解;
    \item 求 $A$ 的广义逆 $A^+$;
    \item 利用广义逆判断方程组 $Ax = b$ 的相容性;
    \item 求 $Ax = b$ 的最小二乘解;
    \item 求 $Ax = b$ 的极小范数最小二乘解。
\end{enumerate}

\end{document}