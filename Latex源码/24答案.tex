\documentclass[12pt,a4paper]{article}
\usepackage{amsmath,amssymb}
\usepackage{ctex}
\usepackage{geometry}
\geometry{left=2cm,right=2cm,top=2cm,bottom=2cm}
\newtheorem{lt}{}
\begin{document}

\begin{center}
\textbf{武汉理工大学研究生考试试卷(A 卷)}

2024--2025 学年 1 学期 \quad 矩阵论(学硕) 课程 \quad 2024 年 12 月 28 日
\end{center}

\section*{一、填空题}
\begin{lt}
$x = (4i, -3i, 12, 0)^T$,由范数定义可得:

$||x||_1 = \displaystyle\sum_{k=1}^4|x_k|=4+3+12+0=19$

$||x||_\infty = \max\ |x_k| = \max \{4, 3, 12, 0\}=12$
\end{lt}




\begin{lt}
先写出$Smith$标准型:

$\lambda I_3 - A = 
\begin{bmatrix} 
\lambda+1 & -1 & 0 \\
 4 & \lambda-3 & 0 \\ 
 -1 & 0 & \lambda-2 
 \end{bmatrix} \rightarrow 
 \begin{bmatrix} 
1 & \lambda+1 & 0 \\
3-\lambda & 4 & 0 \\ 
 0 & -1 & \lambda-2 
 \end{bmatrix}
 \rightarrow 
 \begin{bmatrix} 
1 & \lambda+1 & 0 \\
0 & (\lambda-1)^2 & 0 \\ 
0 & -1 & \lambda-2 
 \end{bmatrix}
   \rightarrow 
\begin{bmatrix} 
1 & 0 & 0 \\
0 & (\lambda-1)^2 & 0 \\ 
0 & -1 & \lambda-2 
\end{bmatrix}
   \rightarrow 
\begin{bmatrix} 
1 & 0 & 0 \\
0 & 1 & -(\lambda-2) \\ 
0 & (\lambda-1)^2 & 0 
\end{bmatrix}
   \rightarrow 
\begin{bmatrix} 
1 & 0 & 0 \\
0 & 1 & 0 \\ 
0 & 0 & (\lambda-1)^2 (\lambda-2)
\end{bmatrix}$.\\
\\

于是可以得到A的极小多项式$m_A(\lambda)=d_3=(\lambda-1)^2 (\lambda-2)=\lambda^3-4\lambda^2+5\lambda-2$,所以根据Caley-Hamilton定理,有$A^3-4A^2+5A-2E=\mathbf{0}$.

因此对$A^5 - 3A^4 + 2A^3 - A^2 + 4A - 4E$利用上式进行带余多项式除法可简化计算:
$$\begin{aligned}A^5 - 3A^4 + 2A^3 - A^2 + 4A - 4E &= (A^3-4A^2+5A-2E)(A^2+A+2E)+(A-2E)\\
&=\mathbf{0}(A^2+A+2E)+(A-2E)=\mathbf{0}+(A-2E)=A-2E
\end{aligned}$$

所以原式$A^5 - 3A^4 + 2A^3 - A^2 + 4A - 4E=A-2E$\\

即本题所求等于$A-2E=\begin{bmatrix}
-1 & 1 & 0 \\ -4 & 3 & 0 \\ 1 & 0 & 2
\end{bmatrix} - 2\begin{bmatrix}
1 & 0 & 0 \\ 0 & 1 & 0 \\ 0 & 0 & 1
\end{bmatrix}=\begin{bmatrix}
-3 & 1 & 0 \\ -4 & 1 & 0 \\ 1 & 0 & 0
\end{bmatrix}$.
\end{lt}





\begin{lt}
$$A(\lambda) = \begin{bmatrix} -\lambda+1 & \lambda^2 & \lambda \\ \lambda & \lambda & -\lambda \\ \lambda^2+1 & \lambda^2 & -\lambda^2 \end{bmatrix}$$

$A(\lambda) \rightarrow  
\begin{bmatrix} -\lambda+1 & \lambda^2 & \lambda \\ 1 & \lambda^2+\lambda & 0 \\ \lambda^2+1 & \lambda^2 & -\lambda^2 \end{bmatrix}
\rightarrow  
\begin{bmatrix} 1 & \lambda^2 & \lambda \\ 1 & \lambda^2+\lambda & 0 \\ 1 & \lambda^2 & -\lambda^2 \end{bmatrix}
\rightarrow  
\begin{bmatrix} 1 & \lambda^2 & \lambda \\ 0 & \lambda & -\lambda \\ 0 & 0 & -(\lambda^2+\lambda) \end{bmatrix}\\
\rightarrow  
\begin{bmatrix} 1 & 0 & 0 \\ 0 & \lambda & -\lambda \\ 0 & 0 & -(\lambda^2+\lambda) \end{bmatrix}
\rightarrow  
\begin{bmatrix} 1 & 0 & 0 \\ 0 & \lambda & 0 \\ 0 & 0 & \lambda^2+\lambda \end{bmatrix}=diag\{1, \lambda, \lambda^2+\lambda \}$.
\end{lt}





\begin{lt}
由已知$A = \begin{bmatrix} 0 & 4 & 1 \\ 1 & 1 & 1 \\ 0 & 3 & 2 \end{bmatrix}$,记$\alpha_1 = \begin{bmatrix} 0 \\ 1 \\ 0\end{bmatrix}$,$\alpha_2 = \begin{bmatrix} 4 \\ 1 \\ 3\end{bmatrix}$,$\alpha_3 = \begin{bmatrix} 1 \\ 1 \\ 2\end{bmatrix}$.

利用Gram-Schmidt正交化方法可得到一组正交基$\{ \beta_1, \beta_2, \beta_3\}$,和其对应的标准单位正交基$\{ \gamma_1, \gamma_2, \gamma_3\}$.

取$\beta_1=\alpha_1 = \begin{bmatrix} 0 \\ 1 \\ 0\end{bmatrix}$,则$\gamma_1 = \displaystyle\frac{\beta_1}{||\beta_1||}= \begin{bmatrix} 0 \\ 1 \\ 0\end{bmatrix}$.\\

取$\beta_2=\alpha_2 -\displaystyle\frac{(\alpha_2, \beta_1)}{||\beta_1||^2}\beta_1= \begin{bmatrix} 4 \\ 1 \\ 3\end{bmatrix} - \frac{1}{1}\begin{bmatrix} 0 \\ 1 \\ 0\end{bmatrix}= \begin{bmatrix} 4 \\ 0 \\ 3\end{bmatrix}$,则$\gamma_2 = \displaystyle\frac{\beta_2}{||\beta_2||}= \begin{bmatrix} 0.8 \\ 0 \\ 0.6\end{bmatrix}$.\\

取$\beta_3=\alpha_3 -\displaystyle\frac{(\alpha_3, \beta_1)}{||\beta_1||^2}\beta_1-\displaystyle\frac{(\alpha_3, \beta_2)}{||\beta_2||^2}\beta_2= \begin{bmatrix} 1 \\ 1 \\ 2\end{bmatrix} - \displaystyle\frac{1}{1}\begin{bmatrix} 0 \\ 1 \\ 0\end{bmatrix}- \frac{10}{25}\begin{bmatrix} 4 \\ 0 \\ 3\end{bmatrix}= \begin{bmatrix} -0.6 \\ 0 \\ 0.8\end{bmatrix}$,则$\gamma_3 = \displaystyle\frac{\beta_3}{||\beta_3||}= \begin{bmatrix} -0.6 \\ 0 \\ 0.8\end{bmatrix}$.\\

根据QR分解定理$A=QR=\begin{bmatrix}\gamma_1 & \gamma_2 & \gamma_3 \end{bmatrix}\begin{bmatrix} ||\beta_1|| & (\alpha_2, \gamma_1) & (\alpha_3, \gamma_1)\\ 0 & ||\beta_2|| & (\alpha_3, \gamma_2)\\ 0 & 0 & ||\beta_3|| \end{bmatrix}$,所以将求得的正交基$\{ \beta_1, \beta_2, \beta_3\}$及其对应的标准单位正交基$\{ \gamma_1, \gamma_2, \gamma_3\}$带入上述公式即可。

$$A = \begin{bmatrix} 0 & 4 & 1 \\ 1 & 1 & 1 \\ 0 & 3 & 2 \end{bmatrix} = \begin{bmatrix} 0 & 0.8 & -06 \\ 1 & 0 & 0 \\ 0 & 0.6 & 0.8 \end{bmatrix}\begin{bmatrix} 1 & 1 & 1 \\ 0 & 5 & 2 \\ 0 & 0 & 1 \end{bmatrix}$$
\end{lt}





\begin{lt}
将$\beta_i$用$\alpha_i$线性表示出来即可。

由$\alpha_1 = \beta_3 - 2\alpha_2$,知$\beta_3=\alpha_1 + 2\alpha_2$.

由$\alpha_2 = \beta_4 - 2\alpha_3$,知$\beta_4 =\alpha_2  + 2\alpha_3$.

由$\beta_2 = \alpha_4 - 2\beta_3$,知$\beta_2 = \alpha_4 - 2\beta_3 = \alpha_4 - 2(\alpha_1 + 2\alpha_2)= - 2\alpha_1 - 4\alpha_2 + \alpha_4$.

由$\beta_1 = \alpha_3 - 2\beta_2$,知$\beta_1 = \alpha_3 - 2\beta_2=\alpha_3-2(- 2\alpha_1 - 4\alpha_2 + \alpha_4)=4\alpha_1+8\alpha_2+\alpha_3-2\alpha_4$.

将上述线性关系用矩阵形式表示即得过渡矩阵:
$$(\beta_1, \beta_2, \beta_3, \beta_4) = (\alpha_1, \alpha_2, \alpha_3, \alpha_4)\begin{bmatrix}
4 & -2 & 1 & 0\\
8 & -4 & 2 & 1\\
1 & 0 & 0 & 2\\
-2 & 1 & 0 & 0
\end{bmatrix}$$
\end{lt}


\section*{二、}
\textbf{解}:只需求出Smith标准型,其它均可由Smith标准型得到。\\

$\lambda E - A \rightarrow
\begin{bmatrix}
\lambda-1& -1 & 1\\
3 & \lambda+3 & -3\\
2 & 2 & \lambda-2\\
\end{bmatrix}
\rightarrow
\begin{bmatrix}
1& -1 & \lambda-1\\
-3 & \lambda+3 & 3\\
\lambda-2 & 2 & 2\\
\end{bmatrix}
\rightarrow
\begin{bmatrix}
1& -1 & \lambda-1\\
0 & \lambda & 3\lambda\\
0 & \lambda & -\lambda^2+3\lambda\\
\end{bmatrix}\\
\rightarrow
\begin{bmatrix}
1& 0 & 0\\
0 & \lambda & 3\lambda\\
0 & \lambda & -\lambda^2+3\lambda\\
\end{bmatrix}
\rightarrow
\begin{bmatrix}
1& 0 & 0\\
0 & \lambda & 0\\
0 & 0 & \lambda^2\\
\end{bmatrix} = diag\{ 1, \lambda, \lambda^2\}
$.\\

由Smith标准型知特征值为$\lambda=0$(三重),行列式因子为:$D_1 = 1, D_2= \lambda, D_3 = \lambda^3$,不变因子为:$d_1 = 1, d_2= \lambda, d_3 = \lambda^2$,初等因子为$\lambda, \lambda^2$。

因此Jordan标准型为$\begin{bmatrix}
0&   &  \\
 & 0 & 1\\
 & 0 & 0\\
\end{bmatrix}= diag\left\{0, \begin{bmatrix}0 & 1\\0 & 0\end{bmatrix} \right\}$.

显然极小多项式$m_A(\lambda) = d_3 = \lambda^2$.





\section*{三、}
\textbf{解}:(1)由已知可直接写出过渡矩阵P,其中P满足$(\beta_1, \beta_2, \beta_3, \beta_4) = (\alpha_1, \alpha_2, \alpha_3, \alpha_4)P$,由于P为下三角阵因此容易得到其逆$P^{-1}$

$$P=\begin{bmatrix}
1 & 0 & 0 & 0\\
-2 & 1 & 0 & 0\\
0 & -1 & 1 & 0\\
1 & -1 & 1 & 1
\end{bmatrix}, 
P^{-1}=\begin{bmatrix}
1 & 0 & 0 & 0\\
2 & 1 & 0 & 0\\
2 & 1 & 1 & 0\\
-1 & 0 & -1 & 1
\end{bmatrix}$$

$$\begin{aligned}P^{-1}AP &=
\begin{bmatrix}
1 & 0 & 0 & 0\\
2 & 1 & 0 & 0\\
2 & 1 & 1 & 0\\
-1 & 0 & -1 & 1
\end{bmatrix}
\begin{bmatrix} 1 & 0 & 2 & 1 \\ -1 & 2 & 1 & 3 \\ 1 & 2 & 5 & 5 \\ 2 & -2 & 1 & -2 \end{bmatrix}
\begin{bmatrix}
1 & 0 & 0 & 0\\
-2 & 1 & 0 & 0\\
0 & -1 & 1 & 0\\
1 & -1 & 1 & 1
\end{bmatrix}\\
&=
\begin{bmatrix} 1 & 0 & 2 & 1 \\ 1 & 2 & 5 & 5 \\ 2 & 4 & 10 & 10 \\ 0 & -4 & -6 & -8 \end{bmatrix}
\begin{bmatrix}
1 & 0 & 0 & 0\\
-2 & 1 & 0 & 0\\
0 & -1 & 1 & 0\\
1 & -1 & 1 & 1
\end{bmatrix}
= \begin{bmatrix} 
2 & -3 & 3 & 1 \\ 
2 & -8 & 10 & 5 \\ 
4 & -16 & 20 & 10 \\ 
0 & 10 & -14 & -8 
\end{bmatrix}\end{aligned}$$
\newpage
(2)$A = \begin{bmatrix} 1 & 0 & 2 & 1 \\ -1 & 2 & 1 & 3 \\ 1 & 2 & 5 & 5 \\ 2 & -2 & 1 & -2 \end{bmatrix}$, rankA = 2,所以$dim$\ Im$T = 2$,根据维数定理,\\ $dim$\ Ker$T = n - dim$\ Im$T = 4-2=2$.

注意到A的第一二列可以线性表示第三四列,所以可写出核空间的基为$\{ \alpha_1 +2\alpha_2 - \alpha_4,\\ 4\alpha_1 +3\alpha_2 -2\alpha_3\}$.




\section*{四、}
\textbf{解}:(1)先求出Smith标准型以得到极小多项式,再利用待定系数法计算即可。\\

$\lambda I-A = \begin{bmatrix}
\lambda+1& 0 &-1\\
-1 & \lambda-2 & 0 \\
4 & 0 & \lambda  -3
\end{bmatrix}
\rightarrow
\begin{bmatrix}
1& 0 &\lambda+1\\
0 & \lambda-2 & -1 \\
3-\lambda & 0 & 4
\end{bmatrix}
\rightarrow
\begin{bmatrix}
1& 0 &\lambda+1\\
0 & \lambda-2 & -1 \\
0 & 0 & (\lambda-1)^2
\end{bmatrix}\\
\rightarrow
\begin{bmatrix}
1& 0 &0\\
0 & \lambda-2 & -1 \\
0 & 0 & (\lambda-1)^2
\end{bmatrix}
\rightarrow
\begin{bmatrix}
1& 0 &0\\
0 & 1 & 0 \\
0 & 0 & (\lambda-1)^2(\lambda-2)
\end{bmatrix}$.\\

所以不难看出极小多项式$m_A(\lambda) = (\lambda-1)^2(\lambda-2)$,deg\ $m_A(\lambda)$=3,所以待定系数的最高次为$3-1=2$次。设$f(\lambda) = a_0+a_1\lambda +a_2\lambda^2$。记$e^{t\lambda} = g(\lambda)$.

考虑$m_A(\lambda) = (\lambda-1)^2(\lambda-2)$,特征值分别为$\lambda = 1$(两重),$\lambda = 2$(一重),带入$f(\lambda)$得:
$$f(1)=a_0+a_1 +a_2=g(1)=e^{t}$$
$$f'(1)=a_1 +2a_2=g'(1)=te^{t}$$
$$f(2)=a_0+2a_1 +4a_2=g(2)=e^{2t}$$

解上述方程组得到系数:
$$a_0= e^{2t}-2te^t$$
$$a_1= -2e^{2t}+(2+3t)e^t$$
$$a_2= e^{2t}-(1+t)e^t$$

$$\begin{aligned}
e^{tA}&= f(A) = a_0E+a_1A +a_2A^2\\
&=(e^{2t}-2te^t)\begin{bmatrix}
1& 0 &0\\
0 & 1 & 0 \\
0 & 0 & 1
\end{bmatrix}+
\left(-2e^{2t}+(2+3t)e^t\right)
\begin{bmatrix}
-1& 0 &1\\
1 & 2 & 0 \\
-4 & 0 & 3
\end{bmatrix}
\\
&+\left(e^{2t}-(1+t)e^t\right)\begin{bmatrix}
-3& 0 &2\\
1 & 4 & 1 \\
-8 & 0 & 5
\end{bmatrix}\\
&=\begin{bmatrix}
-(1-2t)e^t& 0 &te^t\\
-e^{2t}+(1+2t)e^t & e^{2t} & e^{2t}- (1+t)e^t\\
-4te^t & 0 & (2t+1)e^t
\end{bmatrix}
\end{aligned}$$

类似地重复上述操作,可计算出cosA。记$cos\lambda = h(\lambda)$,带入$f(\lambda)$得:
$$f(1)=a_0+a_1 +a_2=h(1)cos1$$
$$f'(1)=a_1 +2a_2=h'(1)=-sin1$$
$$f(2)=a_0+2a_1 +4a_2=h(2)=cos2$$

解上述方程组得到系数:
$$a_0= cos2+2sin1$$
$$a_1=-2cos2+2cos1-3sin1$$
$$a_2= cos2-cos1+sin1$$

$$\begin{aligned}
cosA&= f(A) = a_0E+a_1A +a_2A^2\\
&=(cos2+2sin1)\begin{bmatrix}
1& 0 &0\\
0 & 1 & 0 \\
0 & 0 & 1
\end{bmatrix}+
(-2cos2+2cos1-3sin1)
\begin{bmatrix}
-1& 0 &1\\
1 & 2 & 0 \\
-4 & 0 & 3
\end{bmatrix}
\\
&+(cos2-cos1+sin1)\begin{bmatrix}
-3& 0 &2\\
1 & 4 & 1 \\
-8 & 0 & 5
\end{bmatrix}\\
&=\begin{bmatrix}
cos1+2sin1& 0 &-sin1\\
-cos2+cos1-2sin1 & cos2 & cos2-cos1+sin1\\
4sin1 & 0 & cos1-2sin1
\end{bmatrix}
\end{aligned}$$
\newpage
(2)由于齐次方程组通解为$x(t) = e^{A(t-t_0)x(t_0)}$由$t_0=0$代入得$x(t) = e^{At}x(0)$:
$$\begin{aligned}
x(t)=e^{At}x(0)&=\begin{bmatrix}
-(1-2t)e^t& 0 &te^t\\
-e^{2t}+(1+2t)e^t & e^{2t} & e^{2t}- (1+t)e^t\\
-4te^t & 0 & (2t+1)e^t
\end{bmatrix}\begin{bmatrix} 1 \\ 1 \\ 1 \end{bmatrix}\\
&= \begin{bmatrix} (1-t)e^t \\ e^{2t}+te^t \\ (1-2t)e^t \end{bmatrix}
\end{aligned}$$






\section*{五、}
\textbf{解}:(1)显然.

(2)题目中定义了内积运算$(A, B) = \displaystyle\sum_{i=1}^2 \sum_{j=1}^2 (i+j) a_{ij} b_{ij}$,展开即$(A, B) = 2a_{11}b_{11}+3a_{12}b_{12}+3a_{21}b_{21}+4a_{22}b_{22}$.

由于$W = \left\{ A = \begin{bmatrix} a_{11} & a_{12} \\ a_{21} & a_{22} \end{bmatrix} ; \, a_{11} + a_{22} = 0, \, a_{12} - a_{21} = 0 \right\}$,所以$\forall A \in W, a_{11} =- a_{22},\\ a_{12}= a_{21}$.于是不妨取$\left\{\begin{bmatrix} 1 & 0 \\ 0 & -1 \end{bmatrix}, \begin{bmatrix} 0 & 1 \\ 1 & 0 \end{bmatrix}\right\}$为W的一组基,记作$\{ M, N\}$。

取$X = \begin{bmatrix} a & b \\ c & d \end{bmatrix} \in W^\perp$,由正交补定义有$\forall \alpha \in W, \beta \in W^\perp, (\alpha, \beta) = 0$.因此,$(X,M)=(X,N)=0$:
$$2a-4d=0, \ b+c=0$$\\

不妨取$X_1 = \begin{bmatrix} 2 & 0 \\ 0 & 1 \end{bmatrix}$, $X_2 = \begin{bmatrix} 0 & 1 \\ -1 & 0 \end{bmatrix}$


$$||X_1||^2 = 2 \times 2^2+4\times1^2=12$$
$$||X_2||^2 = 3 \times 1^2+3\times1^2=6$$

于是,\( W \) 的正交补 \( W^\perp \) 的一组标准正交基为:\\

$$\left\{\displaystyle\frac{X_1}{||X_1||}, \displaystyle\frac{X_2}{||X_2||}\right\}=\left\{\displaystyle\frac{1}{2\sqrt{3}}\begin{bmatrix} 2 & 0 \\ 0 & 1 \end{bmatrix}, \displaystyle\frac{1}{\sqrt{6}}\begin{bmatrix} 0 & 1 \\ -1 & 0 \end{bmatrix}\right\}$$

\newpage



\section*{六、(25 分)}
\textbf{解}:(1)注意到矩阵A前两列线性无关,且显然可以线性表示后两列,因此rankA=2。\\

利用矩阵乘法的右乘有$A = \begin{bmatrix} 1 & 2 & 2 &1\\ 1 & 1 & 1 &1\\2& 1 & 1 & 2 \end{bmatrix}=BC= \begin{bmatrix}1&2\\1&1\\2&1\end{bmatrix}
\begin{bmatrix}1&0&0&1\\0&1&1&0\end{bmatrix}$.\\

即得满秩分解A=BC,其中:
$$B=\begin{bmatrix}1&2\\1&1\\2&1\end{bmatrix},
C=\begin{bmatrix}1&0&0&1\\0&1&1&0\end{bmatrix}$$

(2)由于A为$3\times4$阶矩阵,所以Moore-Penrose广义逆$A^+$应为$4\times3$阶矩阵,直接带入公式即可。

$$\begin{aligned}A^+ &= C^H(CC^H)^{-1}(BB^H)^{-1}B^H\\
&= \begin{bmatrix}1&0\\0&1\\0&1\\1&0\end{bmatrix}
\begin{bmatrix}2 & 0\\0 & 2\end{bmatrix}^{-1}
\begin{bmatrix}6 & 5\\5 & 6\end{bmatrix}^{-1}
\begin{bmatrix}1&1&2\\2&1&1\end{bmatrix}\\
&=\displaystyle\frac{1}{22}\begin{bmatrix}1&0\\0&1\\0&1\\1&0\end{bmatrix}
\begin{bmatrix}6 & -5\\-5 & 6\end{bmatrix}
\begin{bmatrix}1&1&2\\2&1&1\end{bmatrix}
=\displaystyle\frac{1}{22}
\begin{bmatrix}
-4&1&7\\
7&1&-4\\
7&1&-4\\
-4&1&7
\end{bmatrix}
\end{aligned}$$

$$A^+=\displaystyle\frac{1}{22}
\begin{bmatrix}
-4&1&7\\
7&1&-4\\
7&1&-4\\
-4&1&7
\end{bmatrix}$$

(3)相容性即验证$AA^+b$是否与b相等即可。

$$AA^+ = \displaystyle\frac{1}{22}
\begin{bmatrix} 1 & 2 & 2 &1\\ 1 & 1 & 1 &1\\2& 1 & 1 & 2 \end{bmatrix}
\begin{bmatrix}
-4&1&7\\
7&1&-4\\
7&1&-4\\
-4&1&7
\end{bmatrix}=\displaystyle\frac{1}{22}
\begin{bmatrix}
20&6&-2\\
6&4&6\\
-2&6&20\\
\end{bmatrix}
$$
$$AA^+b=\displaystyle\frac{1}{22}
\begin{bmatrix}
20&6&-2\\
6&4&6\\
-2&6&20\\
\end{bmatrix}
\begin{bmatrix}1\\1\\1\end{bmatrix}
=\displaystyle\frac{1}{11}\begin{bmatrix}12\\8\\12\end{bmatrix}\neq\begin{bmatrix}1\\1\\1\end{bmatrix}=b$$

显然,方程组 \( Ax = b \) 的相容性不相容。

(5)极小范数最小二乘解为$x^*=A^+b$。

$$x^*=A^+b=\displaystyle\frac{1}{22}
\begin{bmatrix}
-4&1&7\\
7&1&-4\\
7&1&-4\\
-4&1&7
\end{bmatrix}
\begin{bmatrix}1\\1\\1\end{bmatrix}
=\displaystyle\frac{1}{11}
\begin{bmatrix}2\\2\\2\\2\end{bmatrix}$$

(4)最小二乘解为$x=A^+b+(E-A^+A)y$,其中$y\in \mathbf{R^4}$。
$$A^+A=\displaystyle\frac{1}{22}
\begin{bmatrix}
-4&1&7\\
7&1&-4\\
7&1&-4\\
-4&1&7
\end{bmatrix}
\begin{bmatrix} 1 & 2 & 2 &1\\ 1 & 1 & 1 &1\\2& 1 & 1 & 2 \end{bmatrix} 
=\displaystyle\frac{1}{2}
\begin{bmatrix} 1 & 0 & 0 &1\\ 0 & 1 & 1 &0\\0 & 1 & 1 &0\\1 & 0 & 0 &1 \end{bmatrix} 
$$

$$E-A^+A=\displaystyle\frac{1}{2}
\begin{bmatrix} 1 & 0 & 0 &-1\\ 0 & 1 & -1 &0\\0 & -1 & 1 &0\\-1 & 0 & 0 &1 \end{bmatrix} $$

取$y = \begin{bmatrix} a & b & c & d \end{bmatrix}^T$,代入$(E-A^+A)y$得:
$$\begin{aligned}(E-A^+A)y&=\displaystyle\frac{1}{2}
\begin{bmatrix} 1 & 0 & 0 &-1\\ 0 & 1 & -1 &0\\0 & -1 & 1 &0\\-1 & 0 & 0 &1 \end{bmatrix}\begin{bmatrix} a \\ b \\ c \\ d \end{bmatrix}=
\begin{bmatrix} a-d \\ b-c \\ c-b \\ d-a \end{bmatrix}\\
&=(a-d)\begin{bmatrix} 1 \\ 0 \\ 0 \\-1\end{bmatrix}+(b-c)\begin{bmatrix} 0 \\ 1 \\ -1 \\0\end{bmatrix}
\end{aligned}$$

由于y是任意的,因此$(E-A^+A)y=c_1\begin{bmatrix} 1 \\ 0 \\ 0 \\-1\end{bmatrix}+c_2\begin{bmatrix} 0 \\ 1 \\ -1 \\0\end{bmatrix},(c_1,c_2\in \mathbf{R})$

综上,最小二乘解为$x=\displaystyle\frac{1}{11}
\begin{bmatrix}2\\2\\2\\2\end{bmatrix}+c_1\begin{bmatrix} 1 \\ 0 \\ 0 \\-1\end{bmatrix}+c_2\begin{bmatrix} 0 \\ 1 \\ -1 \\0\end{bmatrix},(c_1,c_2\in \mathbf{R})$.
\end{document}